\documentclass[sigconf]{acmart}

\pagestyle{plain}
\usepackage{booktabs} % For formal tables
\usepackage{caption}
\usepackage{subcaption}
\usepackage{listings}
\usepackage{xcolor}

\definecolor{mygreen}{rgb}{0,0.6,0}
\definecolor{mygray}{rgb}{0.5,0.5,0.5}
\definecolor{mymauve}{rgb}{0.58,0,0.82}

\lstset{ %
  backgroundcolor=\color{white},   % choose the background color; you must add \usepackage{color} or \usepackage{xcolor}
  language=C,
  basicstyle=\scriptsize\ttfamily,
  %basicstyle=\footnotesize,        % the size of the fonts that are used for the code
  breakatwhitespace=false,         % sets if automatic breaks should only happen at whitespace
  breaklines=true,                 % sets automatic line breaking
  captionpos=b,                    % sets the caption-position to bottom
  commentstyle=\color{mygreen},    % comment style
  deletekeywords={...},            % if you want to delete keywords from the given language
  escapeinside={\%*}{*)},          % if you want to add LaTeX within your code
  extendedchars=true,              % lets you use non-ASCII characters; for 8-bits encodings only, does not work with UTF-8
  frame=single,                    % adds a frame around the code
  keepspaces=true,                 % keeps spaces in text, useful for keeping indentation of code (possibly needs columns=flexible)
  keywordstyle=[1]\color{blue},       % keyword style
  keywordstyle=[2]\color{mymauve},       % keyword style
  morekeywords=[2]{*, },            % if you want to add more keywords to the set
  numbers=left,                    % where to put the line-numbers; possible values are (none, left, right)
  numbersep=5pt,                   % how far the line-numbers are from the code
  numberstyle=\tiny\color{mygray}, % the style that is used for the line-numbers
  rulecolor=\color{black},         % if not set, the frame-color may be changed on line-breaks within not-black text (e.g. comments (green here))
  showspaces=false,                % show spaces everywhere adding particular underscores; it overrides 'showstringspaces'
  showstringspaces=false,          % underline spaces within strings only
  showtabs=false,                  % show tabs within strings adding particular underscores
  stepnumber=1,                    % the step between two line-numbers. If it's 1, each line will be numbered
  stringstyle=\color{mymauve},     % string literal style
  tabsize=1,                       % sets default tabsize to 2 spaces
  title=\lstname,                   % show the filename of files included with \lstinputlisting; also try caption instead of title
  xleftmargin=10pt,
}

% Copyright
\setcopyright{none}
%\setcopyright{acmcopyright}
%\setcopyright{acmlicensed}
%\setcopyright{rightsretained}
%\setcopyright{usgov}
%\setcopyright{usgovmixed}
%\setcopyright{cagov}
%\setcopyright{cagovmixed}

\begin{document}
\title{Exploiting Task Parallelism at Multiple Hardware Levels of an Accelerator-based Cluster}

%\author{
%Tan Nguyen,
%John Bachan,
%Bryce Lelbach,
%Sam Williams,
%John Shalf,
%David Donofrio, and Cy Chan\\
%Lawrence Berkeley National Laboratory, USA\\
%{\small \{tannguyen, jdbachan, balelbach, swwilliams, jshalf, ddonofrio, cychan\}@lbl.gov}}

\begin{abstract}
Recently heterogeneous architectures have gained traction in HPC. Systems with deep and complex hardware hierarchy and equipped with accelerators each having thousands of cores are increasingly commonplace. On such systems, learning what the application is capable of for the performance optimization purpose is extremely challenging. This paper addresses this non-trivial problem by answering questions such as how many accelerator cores a kernel can scale to, how an irregular application behaves under load balancing granularities, and whether an application benefits from masking/avoiding communication overheads. To this end, we present a solution based on task parallelism, abstracting away many hardware details associated with sophisticated accelerator architectures such as GPUs. We also develop RambutanAcc, a runtime system that can schedule tasks on various hardware levels, ranging from GPU's SMs to compute nodes of a cluster. Experiments on 4 HPC benchmarks show that the flexibility in scheduling tasks results in significant performance gains. 
\end{abstract}

\maketitle

\section{Introduction}
\label{sec:intro}
It is very likely that exascale systems will be armed with powerful accelerators and/or  many-core processors~\cite{ASCR/Exascale/Lethin,  exascaleRoadMap}.
On such systems, performance will heavily depend on the efficient utilization of accelerators/many-core processors, which often requires aggressive, low-level kernel optimizations that destroy portability across systems with different processor architectures.
To handle the performance portability challenge, the programmer must optimize several versions of the same compute kernel.
This problem is challenging especially for domain programmers, given that they must also optimize the code to run on hundreds or thousands of compute nodes.
In this scenario, many runtime systems have been developed to aid the programmer~\cite{legion,physics,mpiacc,mvapich2gpu}.
However, current support for state-of-the-art accelerators is very limited, and there remain significant challenges, such as the development of programming abstractions and fine-grained scheduling mechanisms, as well as efficient communication overlap and dynamic load balancing.  

To address these challenges, we present {\em RambutanAcc}, a task-based runtime for distributed-memory systems that can assist the programmer to increase performance on accelerator-based clusters with modest programming effort.
{\em RambutanAcc} extends the programming interface of {\em Rambutan} \cite{rambutanWebsite}, which provides an interface to construct a fine-grain, dynamic dataflow graph that unfolds as the program executes. 
Each fine-grain task of a {\em RambutanAcc} graph is scheduled to run on a {\em worker}, which can be either a group of CPU cores, an accelerator, or a partition of the accelerator.
Tasks can execute as soon as their {\em true} data dependencies are satisfied, thus avoiding many types of over-synchronization present in other programming models (e.g. barrier, wait\_all, etc.).
Our runtime handles the communication among tasks automatically -- in particular, data required by a task can be produced by other tasks running on CPU or accelerator of a local or remote compute nodes.
The runtime transparently moves data to where the dependent task will execute, and
handles communication in the background so that communication overheads can be overlapped with computation of other runnable tasks.
%\sout{Not only can data move, but tasks can also migrate among different memory address spaces via work stealing.
%Specifically idle workers can find chance to steal tasks from other workers, enabling dynamic load balancing}.

{\em RambutanAcc} currently supports NVIDIA's GPUs and the first generation of Intel's Xeon Phi processor (KNC).
In this paper, we present the GPUs support since GPUs code (e.g. CUDA) is substantially different from conventional code running on the CPU.
Our programming model removes the programming burden of launching CUDA kernels and moving data between the host and GPUs.
Each task is specified as a conventional routine parallelized across a thread team.
%In typical CUDA code, running multiple types of tasks requires launching multiple independent kernels, with little flexibility in when tasks of different types may be executed.
%In contrast, 
{\em RambutanAcc} launches a single {\em persistent CUDA kernel} that can service multiple requests from a task scheduler running on the host, thus avoiding the overhead of multiple kernel launches during execution.
The persistent kernel partitions available SMs (streaming multiprocessors) of a GPUs into workers, which have the flexibility to asynchronously service tasks of different types at the same time.
%This method provides much more flexibility to overlap communication with useful computational work.

{\em RambutanAcc} is also equipped with a communication handler to service communication requests across nodes and between the host and GPUs.
This handler utilizes CUDA streams and works asynchronously with the persistent kernel.
%\footnote{For MIC-based workers, we use Intel's COI (coprocessor offload infrastructure) to implement the on-node handler.}
For off-node requests, {\em RambutanAcc} uses GASNet~\cite{Bonachea:2002:gasnet}, a one-sided communication library % , to implement the communication handler.
that provides non-blocking data transfer and low-latency signaling mechanisms. % , allowing inter-process communication to be overlapped with computations efficiently.

We evaluate {\em RambutanAcc} on up to 16 K80 GPUs using 3 HPC benchmarks: Sparse Cholesky Factorization, Communication-Avoiding Cannon's Matrix Multiplication \cite{25Dcannon}, and 3D Stencil.
The results show that the performance improves significantly due to overlapping communication with computation.
{\em RambutanAcc} also hides communication costs that cannot be avoided.
In addition, for Sparse Cholesky Factorization, having multiple workers on a single GPUs allows compute throughput to be increased at low programming cost. 
We hope this result will encourage further research to develop and tune sparse kernels for co-scheduling fine-grained tasks on GPUs.

The contributions of the paper are three-fold.
\begin{itemize}
\item A task scheduler that supports well fine-grained parallelisms on GPUs without algorithmic change
\item A communication handler that hides various communication overheads at modest programming cost
\item A study of communication overlap in the appearance of a communication avoiding technique 
\end{itemize}

The rest of this paper is organized as follows.
Sec.~\ref{sec:motivation} presents the overview of a hybrid system, which is increasingly popular in practice.
In Sec.~\ref{sec:model}, we present the programming model of {\em RambutanAcc}.
Followed by this section is Sec.~\ref{sec:impl}, which discusses the implementation of the associated runtime system.
Sec.~\ref{sec:results} shows experimental results.
Sec.~\ref{sec:related} presents the related work.
We conclude the paper in Sec.~\ref{sec:conclusion}.



\section{Hybrid Node - a Notable Design Trend}
\label{sec:motivation}
Looking at system designs towards Exascale~\cite{top500}, one will not hesitate to predict that future systems will be based on more powerful compute nodes rather than more nodes~\cite{Shalf:exascaleChallenges}.  
An inevitable consequence of this design trend is that node architecture will be more complex.
Indeed, it is common to see traditional multi-core CPUs augmented with multiple high-end graphics cards such as NVIDIA's GPUs or Intel's first generation Xeon Phi coprocessors (KNC).
Other accelerator architectures such as FPGA and Automata Processor (digital) and Neuromorphic Procecessor (non-digital) have been gaining more traction.
Heterogeneous programming is challenging~\cite{exascaleRoadMap}, mainly because realizing high performance and scalability on accelerators is difficult.
Even when an application performs well on accelerators, load imbalance arises since CPUs and accelerators run at different rates.
Beside the problems above, the programmer has to handle the communication between CPUs and accelerators, e.g. via PCIe, NVLink, etc.
% (see Fig.~\ref{fig:sysArch}).
To avoid these problems, compute nodes can be based on stand-alone many-core processors such as the Intel's second generation Xeon Phi processor (KNL).
However, applications consisting of both task and data parallelism may not suit well to such homogeneous design.
In addition, realizing expected performance on stand-alone many-core processors also requires expert programming skills
to manually manage data movement and locality.



%\begin{figure}[htb]
%\centering
%\begin{subfigure}[b]{0.22\textwidth}
%\includegraphics[width=\textwidth]{figures/SMs.pdf}
%\caption{A GPUs with 4 SMs}
%\label{SMs}
%\end{subfigure}
%\begin{subfigure}[b]{0.2\textwidth}
%\includegraphics[width=\textwidth]{figures/host_accelerator.pdf}
%\caption{Host with accelerators}
%\label{hybrid}
%\end{subfigure}
%\caption{A hybrid node design}
%\label{fig:sysArch}
%\end{figure}


In this paper, we focus on tackling challenges associated with the hybrid CPU-GPUs node design, though our solution can also support the homogeneous case as well as being extended to support other accelerator architectures.


\section{Programming Model}
\label{sec:model}
In this section, we present the {\em RambutanAcc's} task dependency graph representation (or task graph for short) followed by the execution model governing the execution of a task graph program.
Our design goal is to keep the programming API as simple as possible.


\subsection{Task and Data Spaces}
Each {\em RambutanAcc} application works on a directed acyclic graph (DAG) in which vertices are {\em tasks} (see Fig.~\ref{fig:taskGraph}).
A {\em task} is a sequence of statements.
{\em Tasks} are atomic, meaning that when a task is scheduled it runs to completion.
{\em Tasks} are created and destroyed at runtime.
{\em RambutanAcc} separates the task space from data space.
The programmer defines an ownership function to map tasks to partitions of the data space.
Each data partition is associated with an attribute called {\em locale}, which indicates the location of the data (e.g. in GPU's memory).
Task and data do not have to reside in the same physical place, and it is the responsibility of the runtime to move required data to tasks.
The programmer specifies task's inputs and outputs, which are data partitions.
We next describe the full process of defining a task.


\begin{figure}[htb]
\centering
\includegraphics[width=.47\textwidth]{figures/taskGraph.pdf}
\caption{A directed acyclic graph for an iterative solver in two dimensions. The graph partitions the data space and the iteration space. Data space is divided into four partitions, and the color represents the ownership mapping function (tasks to data partitions). Task identifier is a pair of data partition ID and iteration number. Arrows among tasks of different colors represent the data dependencies on ghost cells. Arrows among tasks of the same color represent the task creation order.} 
%Tasks are fireable as soon as data dependencies are satisfied.}
\label{fig:taskGraph}
\end{figure}


\subsection{Defining a task}
For tasks running completely on the host, defining a task includes the specification of inputs and outputs, statements that the task will execute, and new task creation requests.
For tasks running on the accelerator, the programmer has two options for the statement specification phase: (i) each task has some code running on the host to launch the kernel on the accelerator (ii) the runtime directly schedules tasks on accelerator.
In both cases, the runtime handles the communication to make sure that the task is scheduled only when all inputs are available.
%However, in the former case, each task can launch multiple kernels.
In the former case, the programmer has to prepare all arguments needed for the kernel launch and all the required synchronization.
In the latter case, the runtime is responsible for offloading the code and argument information to the accelerator.
With this option, the task can invoke other functions without any special support from hardware (e.g. dynamic parallelism support from CUDA).


\subsection{Execution model}
Among tasks, there are ones that can run without any input from other tasks.
These tasks are called {\em roots}.
In the task graph program described in Fig.~\ref{fig:taskGraph}, tasks ($<$0, 0$>$, $<$1, 0$>$, $<$2, 0$>$, and $<$3, 0$>$ are roots since at the first iteration no nearest neighbor update is required.
Once a {\em root} task finishes its execution, output can be produced.
Now the task notifies the runtime about the availability of the data so the runtime can create new tasks (even remote tasks) and fetch the just produced output data to their locations.
As soon as the data required by a new task have been fetched, that task can become runnable.
Runnable tasks will be scheduled to execute by {\em RambutanAcc} {\em workers}, which can be CPU cores, accelerator, or a portion of the accelerator.
The runtime scheduler periodically checks the availability of {\em workers} to schedule tasks. 
A {\em worker} can be configured with a task buffer having more than one slot so that the scheduler can assign tasks even when the {\em worker} is busy (pre-scheduled). 
This capability allows tasks to be offloaded to {\em workers} on the accelerator at zero cost, since the time to move function arguments from host to accelerator can be overlapped with the execution of another task.

A notable concern when programming on accelerators is that the DRAM capacity on accelerator is often modest compared to host DRAM.
If this is the case, the programmer often decides to keep data on the host and stream only a portion of data to the accelerator at a time to compute before streaming the results back.
Thus, if there are too many tasks created at the same location resulting in so much data being simultaneously fetched to that location, one may run out of memory.
Since {\em RambutanAcc} allocates temporary buffers to perform the fetching work, it has the ability to maintain a reasonable amount of buffer to avoid this problem.


\subsection{Example}
Fig.~\ref{fig:firstProgram} presents some pseudo code to illustrate how a task can be defined to execute on the GPUs.
Lines \#11 to \#17 show how a task registers with the runtime about inputs and outputs.
In this case, a task depends on {\tt Unew} of the previous iteration and ghost cells of other tasks.
The {\em dataMapping} function should swap {\tt Unew} and {\tt Old} for every iteration.
All these data arrays reside in GPU DRAM.
The {\em execute()} function offloads the function that will be executed and required arguments.
System arguments such as thread and block indices are given by the runtime.
%It can be seen that except for the {\em \_\_device\_\_} modifier, the stencil kernel is a generic parallel kernel run by teams of threads (i.e. thread blocks).


\begin{figure}[htp]
\lstinputlisting[caption=]{code/example.c}
\caption{Pseudo code illustrating how to program the iterative solver described in Fig.~\ref{fig:taskGraph} under the {\em RambutanAcc} model. {\em depend\_data()} tells the runtime how to fetch data. 
{\em execute()} tells the persistent kernel which function to invoke.}
{\em dataMapping()} is an application-level function that locates a data partition in the data space. In this application it should swap Uold and Unew for every iteration (recall that the iteration number is encoded in the task name ({\em me})).
\label{fig:firstProgram}
\end{figure}



\section{Implementation}
\label{sec:impl}
In this section we present the implementation details of {\em RambutanAcc}.
At the high level, the runtime is constituted of 3 major modules: task management system, communication handler, and task scheduler as shown in Fig.~\ref{fig:impl}. 
The runtime can be configured to run on a dedicated processor core in order to keep it responsive.

\begin{figure}[htb]
\centering
\includegraphics[width=.49\textwidth]{figures/impl.pdf}
\caption{Runtime implementation}
\label{fig:impl}
\end{figure}

\subsection{Task Management System}
Tasks can be created as soon as input data are produced.
The task management system listens for requests from existing tasks to create new tasks.
A task may own data.
Upon the completion of a task, the task management system may write data back (if necessary) before destroying the task.
If the task executed on accelerator while original data resides in host memory, data will be written back to host asynchronously, overlapping with other tasks' computations.
If original data locate in a different place, the task management system asks the communication handlers to write data back.

\subsection{Communication handlers}
Once a task has been created, it will be pushed to the {\em fetching queue} by the task management system.
The communication handler periodically pulls tasks from this queue and fetches their input data.
Depending on the available memory resource, a certain number of tasks in the {\em fetching queue} will be served at a time.
{\em RambutanAcc} supports both on-node and off-node communication.

Within a single node, we use communication primitives provided by the accelerator's vendor.
In particular, on compute nodes with NVIDIA's GPUs, we use CUDA stream to move data between host and GPUs.
On Intel's KNC based nodes, we use COI (Coprocessor Offload Infrastructure).
To avoid blocking the runtime, we don't use any synchronization routine.
Instead, the communication handler periodically tests the completion of communication activities.

We employ GASNet to implement the off-node communication handler.
The process of fetching data from a remote node is shown in Fig.~\ref{fig:offnode}.
To fetch remote data for a task, the corresponding GASNet process issues a remote procedural call to request data from the process that holds the task owning the data (1).
This procedure looks for the appropriate data at the remote node.
If data resides in accelerator memory, the remote procedure has also to pull data up before sending data (2).
Once the memory transfer from accelerator completes (3), data can be sent to the requester using the one-sided {\em put} operation (4).
Once the remote GASNet process finishes the remote data transfer, it issues a response remote procedural call to push data to the accelerator of the requester (if necessary) and notify the requester when data is available for the task.
All these activities run asynchronously and we use a polling mechanism to avoid blocking the runtime.
The process of writing data back is similar to the fetching process, and it is also handled asynchronously.


\begin{figure}[htb]
\centering
\includegraphics[width=.49\textwidth]{figures/handler.pdf}
\caption{Implementation of the off-node communication handler}
\label{fig:offnode}
\end{figure}

\subsection{Task Scheduler}
Tasks become {\em ready} and pushed in the {\em ready queues} once all of their data have been fetched.
Each GASNet process runs a task scheduler to dispatch {\em ready tasks} to workers. 

\subsubsection{Task Buffer}
To keep workers busy, the scheduler frequently checks the runnable task queue and the status of workers.
However, the task scheduler may not be very responsive since the communication handlers also run on the same processor core.
Thus, each worker has a task buffer with a few slots.
The scheduler fills up these slots while the workers keep popping tasks and execute.
To reduce synchronization overheads we use a lock-free implementation for this single producer-multiple consumers scheme.

\subsubsection{Acc worker using CUDA persistent kernel}
Since the host worker implementation is simple, we now just present the implementation of accelerator workers on GPUs.
A {\em RambutanAcc} program just launches a CUDA kernel to set up workers on GPUs' SMs and execute assigned tasks.
Once the kernel is launched, CUDA thread blocks find out what SM they are mapped to by the CUDA runtime.
Getting this information is possible by inserting PTX code to read a special register that holds SM ID.
As shown in Fig.~\ref{fig:kernel}, we keep only a certain number of thread blocks per SM (this number can be set via an environment variable and the default value is 1).
The reason is that thread blocks run until the program completes and CUDA runtime co-runs only a limited number of thread blocks on the same SM.
The alive thread blocks will be divided into workers.
Each worker is a group of threads blocks, acting as the new CUDA grid for scheduled tasks.

\begin{figure}[htb]
\centering
\includegraphics[width=.35\textwidth]{figures/kernel_init.pdf}
\caption{The persistent kernel selects only a limited number of thread blocks per SM (green color). The default number of thread blocks per SM is 1. 
The alive thread blocks are divided into workers to execute tasks.}
\label{fig:kernel}
\end{figure}

Now each thread block knows which worker it belongs to and where to find tasks to execute.
Once there are ready tasks and an available worker slot, the scheduler offloads corresponding function and arguments to task buffers on the GPUs.
Then the scheduler picks a worker and notifies it by writing a signal to the GPU using the nonblocking copy {\em cudaMemcpyCpyAsync}.
If the worker is busy at the time, it will read the signal  and execute the corresponding task later.
Once the task finishes, the worker notifies the scheduler.
Since a CUDA kernel cannot send data to the host, we have to use UVM (unified virtual memory).




\section{Experimental Results}
\label{sec:results}
\subsection{Experimental Testbed}
In this section, we employ RambutanAcc to study the impact of various code optimizations at different hardware levels of a GPU cluster.
Since an application generally benefits from only a subset of these optimizations, we need more than one application.
To this end, we take four benchmarks commonly used in science and engineering: sparse cholesky factorization, a machine learning based face recognition code, 2.5D Cannon's Matrix Multiplication, and a 3D Stencil.
When studying optimizations on a single GPU, to have apples-to-apples comparisons we always use our own CUDA kernels instead of 3rd party implementations (e.g. cuSPARSE). 
In particular, for Sparse Cholesky Factorization, we develop our own CUDA kernels for $C = \alpha* A * B + \beta C$ where either A, B, and C are all in the {\em sparse} form or A is dense and B and C are sparse.
In these kernels, we perform many optimization techniques such as memory coalescing and spatial tiling.
We also employ a warp based thread reduction implementation that minimizes memory accesses. 
For studies at higher hardware levels, we use vendor provided kernels as much as possible.
For example, in  2.5D Cannon we use the cuBLAS's {\em dgemm} and {\em daxpy} implementations included in the NVIDIA toolkit to perform local matrix multiplication.

For node level optimizations (on single and multiple GPUs of the same compute node), we use up to three K80 GPUs.
A K80 GPU pairs two GPU devices each having 2496 CUDA cores organized into 13 SMs.
The whole K80-based system is equipped with 24GB DRAM with up to 480 GB/s memory bandwidth.
For cluster level optimizations, we run code on K20 GPUs on Titan.
We use the nvcc compiler version 7.0 and sm\_35 capability for CUDA codes and the Intel compiler for codes running on the host.
We use GASNet for communication among GPUs. 


\subsection{Scheduling tasks on SMs of a GPU}

\subsubsection{Sparse Cholesky Factorization}
Sparse Cholesky Factorization A= $LL^T$, where A is a sparse and symmetric positive-definite matrix,
appears in many scientific and engineering problems.
Depending on the sparsity pattern of the input matrix, many sparse representations can be used.
In this paper we employ the CSC (Compressed Sparse Column) format. 
The input matrix is organized as a list of "non-zero" tiles, each including lists of non-zero elements and their row and column indices.
The factorization operation is comprised of three smaller kernels: {\em factor}, {\em solve}, and {\em update}.
These computations on CSC tiles and their data dependencies can be represented by a DAG as shown in Fig.~\ref{fig:cholesky}. 
For very sparse matrices, this DAG may consist of many small tasks.
Thus, this is a perfect application to evaluate the benefit of the fine-grained task scheduling support of the runtime.
We place data on the host's DRAM and execute {\em factor} and {\em solve} on the host's worker.
The compute-intensive {\em update} kernel is executed on the GPU workers.
{\em RambutanAcc} automatically streams data required by this kernel to the GPU's DRAM and streams the results back.

\begin{figure}[htb]
\centering
\includegraphics[width=.3\textwidth]{figures/cholesky.pdf}
\caption{Cholesky factorization DAG consisting of three types of tasks: F (Factor), S (Solve), U (update). Each task is associated with a partition of the input matrix called a {\em tile}. Arrows presesent data dependencies between tasks of different types or of the same task type but on different tiles.}
\label{fig:cholesky}
\end{figure}


An interesting path to explore is the tradeoff between coarse and fine-grained scheduling policies.  
For sparse cholesky, the matrix is represented by many small CSC tiles.\samW{the previous sentence repeats the one}
Thus, even a small problem size can result in many tasks.
Fig.~\ref{fig:coarseFine} shows results of sparse cholesky under two scheduling policies.
It can be seen that fine-grained scheduling policy outperforms the coarse-grained one.
This can be explained as follows.
Each CSC tile is very small (e.g. 32$\times$32), making it hard to map computations efficiently to many CUDA cores.
Thus, it may not be posible to scale a task to all available SMs of a GPUs.
We observe that for many input matrices tasks run more efficiently after reducing the number of SMs per worker by a factor of 2$\times$ or 4$\times$.
Since there are many tasks that can be runnable at a time, the scheduler can keep all SMs busy at very small overhead.
Fig~\ref{fig:nWorkers} shows the optimal number of workers on a K80 GPU for Sparse Cholesky Factorization when the degree of sparsity of the input matrix varies.
If the sparse matrix is filled with many tasks we have more parallelism, allowing us to configure the GPUs with more workers.

The lesson learn from this study is that fine-grained scheduling can be very helpful if we have a DAG with many small tasks, which can not run well on the whole GPU-based system.
This is an important observation since sparse representation is very common in practice.
{\em RambutanAcc} runtime supports fine-grained task scheduling, a simple yet powerful solution to this problem.
The programmer can obtain high compute throughput on GPUs without complicating the application algorithm.

\begin{figure}[htb]
\centering
\begin{subfigure}{0.23\textwidth}
\includegraphics[width=\textwidth]{figures/choleskyScheResults.pdf}
\caption{Tile size 32$\times$32}
\label{choleskySche}
\end{subfigure}
\begin{subfigure}{0.23\textwidth}
\includegraphics[width=\textwidth]{figures/nWorkers.pdf}
\caption{The optimal number of workers/GPUs}
\label{fig:nWorkers}
\end{subfigure}
\caption{Coarse vs. fine-grained Scheduling\samW{note left vs. right;  for left, what is the sparsity???  where does the matrix come from???}}
\label{fig:coarseFine}
\end{figure}


\subsubsection{Viola-Jones Face Recognition}
\samW{is there a citation for this code/algorithm}
We next study the impact of dynamic task scheduling in balancing the workload among SMs of a GPU.
For this study we use the Viola-Jones face detection kernel, an important module in many applications such as security surveillance.
The Viola-Jones face detection algorithm detects faces by scanning a regtangular window of pixels over the image where it looks for features of a human face. 
If a window contains a significant number of these features, it is considered to be a face. 
Since face size varies, the window is scaled a number of times and the scanning process is repeated. 
To reduce the number of features that each window needs to check, the window passes through a number of different stages. 
Early stages have fewer features to check and are easier to pass whereas later stages have more features and are more selective. 
At each stage, the calculations of features are accumulated and, if this accumulated value does not pass the threshold, the stage is failed and the current window is considered to not contain a face. 

For this application, it is straightforward to exploit parallelism among search windows.
Each CUDA thread block is responsible for a fixed number of windows, which will be further distributed to threads within the thread block.
We call this code variant {\em CUDA-Basic}.
Since the number of instructions per window depends on the input, the impact of thread divergion is expected to be significant.
Thus, we also employ a {\em CUDA-Static Warp Scheduling} version which allows 32 threads in a warp to share a window (and thus they perform the same number of instructions).
Porting this code on RambutanAcc, we expect to improve the performance further by balancing the workload among these warps.
Finally, we run a hand-optimized code variant, which embeds the task scheduler into the kernel code.
This code has all the capabilities that RambutanAcc can, but at lower cost since the task scheduler is specialized and runs on the GPU.

\begin{figure}[htb]
\centering
\includegraphics[width=0.35\textwidth]{figures/faceRecognition.pdf}
\caption{Balance the face searching\samW{what is FPS?  I assume up is good... i.e. FPS is a throughput rather than time metric}}
\label{faceRecognition}
\end{figure}

Fig.~\ref{faceRecognition} shows the performance of all code variants.
{\em CUDA-Basic} performs poorly as expected.
Under this straightforward strategy, only one or a few threads among 32 threads of the same warp reach to late stages while the rest stay idle.\samW{previous sentence is unclrear...  Are you saying the naive code has a thread divergence problem?}
This explains why {\em CUDA-Static Warp Scheduling} improves the performance significantly.
However, there remains significant load imbalance among warps.
With RambutanAcc, windows are distributed to workers dynamically.
Specifically, we configure the runtime with 26 workers (two workers per SM), each a CUDA thread block. 
The scheduler running on the host keeps assigning blocks of windows to these workers.
In order to hide the scheduling latency, we configure the task buffer of each worker with multiple slots.
While the worker is processing the current window block, the scheduler can offload another block to the remaining slots.
In this experiment, each block takes about 50$\mu$s to finish, whereas the offloading cost is around 10$\mu$s.
Thus, configuring the task buffer with two slots is sufficient.
Fig.~\ref{faceRecognition} shows that RambutanAcc speeds up {\em CUDA-Static Warp Scheduling} by 1.35$\times$.
All the performance improvement can be attributed to the capability of balancing computationgs among SMs of the GPU.
The {\em CUDA-Hand Optimized} version runs even faster.
The additional performance improvement is due to reducing scheduling overhead by embedding the scheduler to the application source code.
\samW{where did the CUDA versions of the code come from?  Did you write them all?  if not, cite source}


\subsection{Scheduling tasks on GPUs of the same node}

\subsubsection{3D Stencil}
\samW{3D stencil is an ill-defined term... is this a 7-point, constant coefficient laplacian?  Are you doing a smoother with a RHS?  are you assuming periodic BCs?  or constant(node/vertex centerd) boundaries?}
On multiple GPUs we pick {\em 3D Stencil}, an iterative solver for Laplace's equation in three dimensions.
{\em 3D Stencil} iterates over a 3D mesh, updating data elements using values from six nearest neighbors.
The DAG for this application is similar to that shown earlier in Fig.~\ref{fig:taskGraph}, except for the number of dimensions.
In particular, each task is associated with a data partition with up to six ghost cells.
A task can be run when the previous iteration on this data partition finishes and it pulls all the needed ghost cells from neighboring tasks.
3D stencil is a memory bandwidth bound application. 
Thus using GPUs can boost up the performance significantly.
It is interesting to determine if {\em RambutanAcc} can improve the performance further by  hiding communication overheads.

\begin{figure*}[htb]
\begin{subfigure}[b]{0.47\textwidth}
\centering
\includegraphics[width=0.9\textwidth]{figures/stencil_single_node_tida.pdf}
\caption{A compute node with 6 Kepler GPUs (3 K80 cards) connected via a shared PCIe bus}
\label{stencil_single_node_tida}
\end{subfigure}
\begin{subfigure}[b]{0.47\textwidth}
\centering
\includegraphics[width=0.9\textwidth]{figures/stencil_single_node_summit.pdf}
\label{stencil_single_node_summit}
\caption{A compute node with 4 Pascal (P100) GPUs. GPUs \#1 and \#2, \#3 and \#4 are paired togethor via an NVlink interconnect.}
\end{subfigure}
\caption{Strong scaling study on single compute nodes. Problem size $512^3$.
%\samW{I suggest the x-axis be GPU devices (2,4,6).  It also makes sense for there to be a single device (no communication) data point.}
}
\end{figure*}




\subsubsection{2.5D Cannon Matrix Multiply}
Although sparse representations are widely used in practice, dense matrix operations also have a significant share in many scientific and engineering areas.
As a result, we employ a dense matrix multiplication operation $C = \alpha* A * B + \beta C$ to evaluate our runtime.
This is a compute bound application, and the GPU architecture is very well suited for the computation. 
There are many algorithms for the matrix multiply operation, and we use a well-known extension of the standard 2D Cannon's algorithm called {\em Communication Avoiding} or 2.5D Cannon~\cite{25Dcannon}. 
Under the original 2D Cannon's algorithm, the available tasks are organized into a {\em T=PxP} mesh, partitioning each of the three matrices A, B, and C into blocks.
These partitions are first aligned using a skewing operation.
The algorithm then performs P computation steps accumulating the C partition using the rotated A and B partitions.
The communication avoiding algorithm shown in Fig.~\ref{fig:25DCannon} replicates the input matrices by a factor of L using an additional task dimension.
The algorithm broadcasts input data to layers in this dimension to compute the traditional Cannon with T/$\sqrt(L^3)$ steps then reduces the results back to the first layer.

\begin{figure*}[htb]
\centering
\begin{subfigure}[b]{0.45\textwidth}
\includegraphics[width=\textwidth]{figures/cannon0.pdf}
\caption{A and B dependencies on each replication layer}
\label{deps}
\end{subfigure}
\begin{subfigure}[b]{0.37\textwidth}
\includegraphics[width=\textwidth]{figures/cannon1.pdf}
\caption{Results are reduced to the original layer}
\label{dataspace}
\end{subfigure}
\caption{Computing $C= \alpha A * B + \beta C$ using the 2.5D Cannon's matrix multiplication algorithm given the input matrices are already replicated and aligned. 
The DAG partitions matrix C and the step space of the algorithm.
Task Id is a triple where the first two numbers represent the coordinates of a C partition and the last is the step number of the algorithm. 
Fig.~\ref{deps} shows two subsets of the graph to illustrate two types of data dependecies required to compute C.}
\label{fig:25DCannon}
\end{figure*}


\begin{figure}[htb]
\centering
\includegraphics[width=0.49\textwidth]{figures/cannon_tida.pdf}
\caption{2.5 Cannon on two K80s (four GPU devices)}
\label{cannon_onnode}
\end{figure}




\subsection{Scheduling tasks at the cluster level}
%\samW{is this a missing subsection? or is communication hiding supposed to be a subsubsection???}


\begin{figure}[htb]
\centering
\includegraphics[width=0.49\textwidth]{figures/stencil_multiple_nodes_summit.pdf}
\caption{Stencil on multiple nodes of Summit}
\label{stencil_multiple_nodes}
\end{figure}



%\subsubsection{Communication hiding}
%We now extend the experiment to multiple GPUs.
%In this experiment, we evaluate the benefit of hiding the communication overheads among GPUs.
%To this end, we configure the runtime in two modes: {\em no overlap} and {\em overlap}.
%The former uses blocking CUDA memory copy routines to transfer data between host and GPUs while the latter uses non-blocking variants.
%Since the fine-grained scheduler is not compatible with blocking mode (the persistent kernel runs to completion while blocking routines can't proceed until all previously submitted CUDA kernels have completed), we use the coarse-grained scheduling policy for both the blocking and non-blocking modes.

%Fig.~\ref{overlap} shows results of three applications under two communication modes.
%In this study, we do not replicate the input matrices of the 2.5D Cannon's algorithm because the communication avoiding technique may interfere with the communication overlap.
%We will study this interference later in Sec~\ref{subsec:CAvsOlap}.
%It can be seen in Fig.~\ref{overlap} that on three applications {\em overlap} always outperforms {\em no overlap}.
%In Cholesky, we place data on the host and stream them to GPUs to perform the compute-intensive {\em update} kernel.
%Thus, even on one GPU, communication arises.
%\footnote{Although we do not show results of 2.5D Cannon and 3D Stencil on one GPU, it's worth noting that computing on one GPU doesn't incur communication cost since we initially place data on the GPU.
%As a result, we do not observe performance improvement when running these two applications on only one GPU.}
%On multiple GPUs, we realize notable performance improvement via overlapping communcation with computation.
%The overall time reduction is 10\% more or less.
%For Stencil, however, we see a higher speedup (up to 1.85$\times$) due to the following reason.
%At a small scale 1D decomposition works best since it does not require the costly packing and unpacking operations.
%However, with a 1D decomposition scheme the amount of communication does not decrease as the number of GPUs increases.
%Thus, the more GPUs the higher communication relative to computation, resulting in a better improvement due to overlap.
%Unlike 3D Stencil, experiments on the other two applications use a 2D decomposition scheme.
%Thus, the communication over computation ratio does not change much as the number of GPU increases.

%\begin{figure*}[htb]
%\centering
%\includegraphics[width=0.9\textwidth]{figures/overlap.pdf}
%\caption{Hiding communication automatically via overlap
%\samW{How much potential was there for overlap in the first place?}
%%\samW{When strong scaling, there is a region where it might be viable.}
%\samW{When weak scaling, you may always be outside the range where overlap is viable.}
%}
%\label{overlap}
%\end{figure*}

%\begin{figure}[htb]
%\centering
%\includegraphics[width=0.49\textwidth]{figures/CA_4096.pdf}
%\caption{2.5D Cannon on 16 GPUs using small matrices (N=4096). The communication avoiding technique results in many task configurations. For good configurations (e.g. \{64, 4\}), there is not much room for the communication overlap. However, for poor configurations (e.g. \{1024, 4\}) the overlap technique does a good job in further increasing the performance}
%\label{CA_4096}
%\end{figure}

%\subsubsection{Interference between communication avoiding and hiding}
%\label{subsec:CAvsOlap}
%Now let's study the behavior of {\em overlap} when the communication avoiding technique is enabled.
%Fig.~\ref{CA_4096} shows the run time of the 2.5D Cannon program when performing matrix multiplication on 16 GPUs.
%We can see that replicating matrices substantially reduces the run time.
%Notable examples are  \{64, 4\} compared to \{64, 1\} and \{256, 4\} compared to \{256, 1\}.
%If most of the data communication can be avoided, there is not much left to hide.
%However, the number of these optimal replication configurations is very small compared to the combination of task and replication spaces.
%As a result, the programmer may need to brute force many potential configurations to find the best one.
%This requirement is costly and time consuming.
%Luckily the overlapping technique can work with communication avoiding within the same application.
%Thus, if the programmer does not pick the best replication configuration, he/she can rely on the communication overlap to realize comparable performance.
%For example, the {\em overlap} performance on configurations \{128, 2\}, \{256, 4\}, or the most wanted \{64, 1\} is  acceptable. 
%We can see that there are many of such configurations, allowing the programmer to guess one easily.



\section{Related work}
\label{sec:related}
\input{related}

\section{Conclusion}
\label{sec:conclusion}
We have presented a task-based programming model and runtime system for accelerator-based clusters.
Experimental results on four benchmarks show that with the new code representation the programmer can realize better performance.
The performance improvement comes from increasing throughput of processing fine-grained tasks on each individual GPU, dynamic load balancing, direct communication among GPUs and/or hiding communication overheads.
Our solution offers a simple interface so the programmer can study these benefits without the needs of redesigning the algorithm and aggressively restructuring the source code as the system architecture evolves.
We deem that our paper not only has impact on application development, but it can also initiate further research on HPC libraries such as developing and tuning cuSPARSE routines for running on the same GPUs. 



\section*{Acknowledgment}
All authors from LBNL were supported by the Office of Advanced Scientific Computing Research in the Department of Energy Office of Science under contract number DE-AC02-05CH11231.
RambutanAcc and application codes were developed on TiDA, a workstation housed at LBNL.
All GPUs on TiDA were provided by NVIDIA.
Experiments were conducted on Comet at SDSC.
We would like to thank Paul Hargrove for his great help on GASNet.
We also thank Yifeng Cui for time allocations on the Comet system.
Tan Nguyen was a fellow of the Vietnam Education Foundation while conducting this research.

\bibliographystyle{ACM-Reference-Format}
\bibliography{ref} 


\end{document}
